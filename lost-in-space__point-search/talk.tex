\documentclass{beamer}

\mode<presentation>
{
  \usetheme{default}
  \setbeamertemplate{navigation symbols}{}
  \setbeamertemplate{footline}[frame number]
  \setbeamertemplate{items}[circle]
  \usecolortheme{seahorse}
}

\usepackage[english]{babel}
\usepackage[utf8]{inputenc}
\usepackage{times}
\usepackage[T1]{fontenc}

\title[Lost!] % (optional, only for long titles)
{Lost in Space}
\subtitle{Binary search trees beyond dimension one}

\author[Abrahamson]
{Jeff Abrahamson}
\institute[Google]{Google, Inc.}

\date[Big-O Meetup]
{London Big-O Meetup, 23 May 2014}

\subject{kd-trees, range trees, and other generalizations of BST's}
% This is only inserted into the PDF information catalog. Can be left
% out.

% Delete this, if you do not want the table of contents to pop up at
% the beginning of each subsection:
\AtBeginSubsection[]
{
  \begin{frame}<beamer>{Outline}
    \tableofcontents[currentsection,currentsubsection]
  \end{frame}
}

% If you wish to uncover everything in a step-wise fashion, uncomment
% the following command: 
%\beamerdefaultoverlayspecification{<+->}

\begin{document}

% \frame{\titlepage}

\begin{frame}
  \titlepage
\end{frame}

\begin{frame}
  \frametitle{Outline}
  \tableofcontents[pausesections]
\end{frame}

\section{$\mathbb{R}^1$}
\subsection{BST}

\begin{frame}
  \frametitle{Binary Search Trees}
  \begin{itemize}
  \item Is $p\in S$?
  \item Given $x$, what is closest point $p\in S$?
  \item Find $\{x\,|\,x\in (p_1,p_2)\}$.
  \item Given $\delta$, find $\{x\,|\,\mathrm{d}(x,p) < \delta\}$.
  \end{itemize}
  \pause
  \bigskip
  \textcolor{blue}{This is easy in $\mathbb{R}^1$.  What about $\mathbb{R}^d$?}
\end{frame}
\end{document}

\section{$\mathbb{R}^d$}
\subsection{Quadtrees}

\begin{frame}
  \dots
\end{frame}

\subsection{kd-trees}

\begin{frame}
  \dots
\end{frame}

\subsection{Range trees}

\begin{frame}
  \dots
\end{frame}

\subsection{More}

\begin{frame}
  \begin{itemize}
  \item Voronoi diagrams
  \item \dots
  \item database range searches
  \item curse of dimensionality, volume of unit cube $\plusminus\epsilon$, number of cubes covering unit cube, distance from $(0,0,\ldots,0)$ to $(1,1,\ldots,1)$.
  \item What if we want to insert or delete points?
  \end{itemize}
\end{frame}

\section*{Summary}

\begin{frame}
  \dots
\end{frame}

\end{document}
