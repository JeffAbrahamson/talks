\input ../talk-header.tex
\title
{ML Week}
\subtitle{0x01 \hspace{2mm}  Introduction}

% If you wish to uncover everything in a step-wise fashion, uncomment
% the following command: 
%\beamerdefaultoverlayspecification{<+->}

\begin{document}

\begin{frame}
  \titlepage
\end{frame}


\begin{frame}
  \phrase{Definition}
\end{frame}

\begin{frame}
  \phrase{Supervised}
\end{frame}

\begin{frame}
  \phrase{Unsupervised}
\end{frame}

\begin{frame}
  \phrase{Reinforcement}
\end{frame}

\begin{frame}
  \phrase{Curse of Dimensionality}
\end{frame}

\begin{frame}
  \frametitle{Addition rule: independent events}
  \phrase{$\Pr(A\cup B) = \Pr(A) + \Pr(B)$}
\end{frame}

\begin{frame}
  \frametitle{Addition rule: dependent events}
  \phrase{$\Pr(A\cup B) = \Pr(A) + \Pr(B) - \Pr(A\cap B)$}
\end{frame}

\begin{frame}
  \frametitle{Multiplication rule: independent events}
  \phrase{\blue{$\Pr(A\cap B) = \Pr(A) \Pr(B)$}}
\end{frame}

\begin{frame}
  \frametitle{Multiplication rule: dependent events}
  \phrase{$\Pr(A\cap B) = \Pr(A\mid B) \Pr(B)$}
\end{frame}

\begin{frame}
  \frametitle{Conditional probability}
  \phrase{$\Pr(A\mid B) = \frac{\Pr(A\cap B)}{\Pr(B)}$}
\end{frame}

\begin{frame}
  \frametitle{Conditional probability}

  \begin{mphrase}
    \cup_i A_i = A \quad\land\quad A_i \cap A_j = \emptyset \implies
  \end{mphrase}

  \begin{mphrase}
    P(A_1 \mid B) = \frac{\Pr(B\mid A_1) \Pr(A_1)}{\sum_i \Pr(B\mid
      A_1)\Pr(A_1) + \dots + \Pr(B\mid A_k)\Pr(A_k)}
  \end{mphrase}
\end{frame}

\begin{frame}
  \frametitle{What is Statistics}
  
  \begin{enumerate}
  \item<1-> Identify a question or problem.
  \item<1-> Collect relevant data on the topic.
  \item<1-> Analyze the data.
  \item<1-> Form a conclusion.
  \end{enumerate}
  \only<2>{Sadly, sometimes people forget 1.}
  \only<3->{Statistics is about making 2--4 efficient, rigorous, and meaningful.}
  \only<3->{\vfill\prevwork{\textit{OpenIntro Statistics},
      2nd edition, D.~Diez, C.~Barr, M.~Çetinkaya-Rundel, 2013.}}

\end{frame}

\begin{frame}
  \frametitle{What is data science?}

  (Exercise: Is this the same question as the last slide?)

  \only<1>{
    \begin{enumerate}
    \item Define the question of interest
    \item Get the data
    \item Clean the data
    \item Explore the data
    \item Fit statistical models
    \item Communicate the results
    \item Make your analysis reproducible
    \end{enumerate}
  }

  \only<2>{
    \begin{enumerate}
    \item Define the question of interest
    \item Get the data
    \item Clean the data
    \item \purple{Explore the data}
    \item \purple{Fit statistical models}
    \item Communicate the results
    \item Make your analysis reproducible
    \end{enumerate}

    \blue{What the public thinks.}
  }

  \only<3>{
    \begin{enumerate}
    \item Define the question of interest
    \item \purple{Get the data}
    \item \purple{Clean the data}
    \item Explore the data
    \item Fit statistical models
    \item \purple{Communicate the results}
    \item \purple{Make your analysis reproducible}
    \end{enumerate}

    \blue{Where we spend most of our time.}
  }

  \only<4>{
    \begin{enumerate}
    \item \purple{Define the question of interest}
    \item Get the data
    \item Clean the data
    \item Explore the data
    \item Fit statistical models
    \item Communicate the results
    \item Make your analysis reproducible
    \end{enumerate}

    \blue{The easiest part to forget.}
  }

  \only<5>{\vfill\prevwork{\url{http://simplystatistics.org/2015/03/17/data-science-done-well-looks-easy-and-that-is-a-big-problem-for-data-scientists/}}}
\end{frame}

\begin{frame}
  \frametitle{Anecdote}

  Some properties of anecdote:
  
  \begin{itemize}
  \item is data
  \item haphazardly collected
  \item is generally not representative
  \item sometimes result of selective retention
  \item does not accumulate to be representative
  \item might be true (by chance)
  \item is ok to use as hypothesis, but be clear that hypothesis is anecdote
  \end{itemize}
\end{frame}

\begin{frame}
  \frametitle{Study Types}

  \begin{itemize}
  \item Observational
  \item Experimental
  \end{itemize}
  
  \only<2>{
    What can go wrong?
    \begin{itemize}
    \item Forgetting that association $\ne$ causation
    \item Not random
    \item Confounding variables
    \end{itemize}
  }
\end{frame}

\begin{frame}
  \frametitle{Variable types}
  \cimg{variable-types.png}
\end{frame}

\begin{frame}
  \cimggg{bias-variance.png}
\end{frame}

\begin{frame}
  \frametitle{Mean}

  \begin{itemize}
  \item Weighted and unweighted
  \item Centroid to physicists
  \end{itemize}

  \only<2-4> {
    \cimgg{teeter-totter.png}
  }
  \only<3>{
    \begin{displaymath}
      \mu = E(X) = \sum w_i x_i = \mathbf{w\cdot x}
    \end{displaymath}

    \vfill
    \prevwork{\url{http://telescopes.stardate.org/images/research/teeter-totter/TT4.gif}}
  }
  \only<4>{
    \begin{displaymath}
      \mu = E(X) = \sum \Pr(X=x_i) x_i
    \end{displaymath}

    \vfill
    \prevwork{\url{http://telescopes.stardate.org/images/research/teeter-totter/TT4.gif}}
  }
  \only<5>{
    \begin{displaymath}
      \mu = E(X) = \int xf(x) \D{x}
    \end{displaymath}

    \vfill
    \prevwork{\url{http://telescopes.stardate.org/images/research/teeter-totter/TT4.gif}}
  }
  \only<6>{
    \cimgg{centroid-hanging-discrete.png}
  }
  \only<7>{
    \cimgg{centroid-balance-continuous.png}
  }
\end{frame}

\begin{frame}
  \frametitle{Population statistics}

  \soloo{1}{\textbf{Deviation} is distance from mean.}
  \soloo{2}{\textbf{Variance} is mean square of deviations}
  \soloo{3}{\textbf{Standard deviation} is square root of variance}

  \only<4>{
    \vfill
    \begin{displaymath}
      s^2 = \frac{(\overline x - x_1)^2 + \dotsb (\overline x - x_n)^2}{n-1}
    \end{displaymath}
  }
  \only<5>{
    \vfill
    \begin{displaymath}
      \sigma^2 = \frac{(\overline x - x_1)^2 + \dotsb (\overline x - x_n)^2}{n}
    \end{displaymath}
  }
  \only<6>{
    \vfill
    \begin{displaymath}
      \mbox{Var}(X) = \sigma^2 = (\overline x - x_1)^2 \Pr(X=x_1) + \dotsb
        (\overline x - x_n)^2 \Pr(X=x_n)
    \end{displaymath}
  }
\end{frame}

\begin{frame}
  \frametitle{Words}

  Given a distribution $X$, the

  \only<1> {
    \textbf{probability distribution function (pdf)} (continuous) or
    \textbf{probability mass function (pmf)} is the probability that
    the variate has value $x$:
    
    \begin{displaymath}
      \Pr(a\le X \le b)
    \end{displaymath}
    \centerline{or}
    \begin{displaymath}
      \Pr(X=a)
    \end{displaymath}
  }
  
  \only<2> {
    \textbf{cumulative probability function (cdf)} is the probability
    that the variate is less than $x$:

    \begin{displaymath}
      \Pr(X\le x) = \int_{-\infty}^x \text{pdf}(x) \D{x} \,\text{ or }\, \sum_{i\le x} \Pr(X = x)
    \end{displaymath}
  }
  
  \only<3> {
    \textbf{percent point function (ppf)} is the inverse of the cdf.
    Given a probability, what's $x$?
    Also called the \textbf{inverse distribution}.
  }
  
  \only<4> {
    \textbf{survival function (sf)} is the probability that the
    variate takes a value greater than $x$:

    \begin{displaymath}
      \text{ss}(x) = \Pr(X > x) = 1-\text{cdf}(x)
    \end{displaymath}
  }
  
  \only<5> {
    \textbf{inverse survival function (isf)} is the inverse of the
    survival function:

    \begin{displaymath}
      \text{isf}(\alpha) = \text{ppf}(1-\alpha)
    \end{displaymath}
  }
\end{frame}

\begin{frame}
  \cimgh{boxplot-vs-pdf.png}
\end{frame}

\begin{frame}
  \frametitle{Evaluating Normal Approximations}

  \only<1>{Easy technique 1: visually compare to normal plot.

    \cimg{normal-plot.png}
  }

  \only<2> {Easy technique 2: normal probability plot.

    \cimggg{normal-quantile.png}

    Also known as a quantile-quantile plot.
  }
\end{frame}

\begin{frame}
  \begin{displaymath}
    \overline{x}\ne\mu
  \end{displaymath}
\end{frame}

\begin{frame}
  \frametitle{Inference Concepts}

  \only<1->{\textbf{Running mean.}  Sequence of partial sums (divided
    by number in sum).
  }

  \only<2>{
    \bigskip
    \cimgg{running-mean.png}
  }

  \only<3->{\textbf{Sampling variation.}  Change of $\overline x$ from
    one sample to the next.
  }

  \only<4->{\textbf{Sampling distribution.}  The distribution of
    possible point samples of a fixed size from a given population.
  }
\end{frame}

\begin{frame}
  \frametitle{Sampling distribution}

  \cimg{sampling-distribution.png}
\end{frame}

\begin{frame}
  \frametitle{Confidence intervals}

  Sample $n$ points, choose an interval around the sample mean.

  A 95\% confidence interval means if we sample repeatedly, about 95\%
  of the samples will contain the population mean.

  \only<2>{
    \cimg{confidence-intervals.png}
  }

  \only<3>{
    \cimg{confidence-intervals-2.png}
  }
\end{frame}


%%%%%%%%%%%%%%%%%%%%%%%%%%%%%%%%%%%%%%%%%%%%%%%%%%%%%%%%%%%%%%%%%%%%%%
%\talksection{Break}

\begin{frame}
  \frametitle{Questions?}
  \centerline{\large\url{purple.com/talk-feedback}}
\end{frame}

\end{document}
