\input ../notes-header.tex

\begin{document}

\notetitle{08}{PCA}

L'analyse en composantes principales (ACP)

Also known as\dots
\begin{itemize}
\item Discrete Kosambi-Karhunen–Loève transform (KLT) (signal processing)
\item Hotelling transform (multivariate quality control)
\item Proper orthogonal decomposition (POD) (ME)
\item Singular value decomposition (SVD), Eigenvalue decomposition (EVD) (linear algebra)
\item Etc.
\end{itemize}

Uses
\begin{itemize}
\item Exploratory data analysis
\item Compression
\end{itemize}

Think of it as fitting an $n$-dimensional ellipsoid to the data.
\begin{itemize}
\item Each access is a principle component
\item Think about linear transformations as mapping basis vectors
\end{itemize}

Eigenfaces
\begin{itemize}
\item Sirovich and Kirby (1987)
\begin{itemize}
\item searching for low-dimensional representation of face images
\item PCA to get set of basis vectors (eigenpictures)
\end{itemize}
\item Turk and Pentland (1991)
\begin{itemize}
\item Actually worked on computers of the time
\item PCA on face space was too expensive
\item So compute eigenvectors of a covariance matrix instead
\item Matrices sized by the number of images rather than number of pixels
\end{itemize}
\item Each component represents some abstract features
\item Also used in
\begin{itemize}
\item handwriting recognition
\item lip reading
\item voice recognition
\item sign language/gesture interpretation
\item medical image analysis
\end{itemize}
\item Advantages
\begin{itemize}
\item Easy, relatively inexpensive
\item Preprocessing more expensive than recognition
\item A reasonably large database is possible
\end{itemize}
\item Disadvantages
\begin{itemize}
\item Sensitive to lighting, scale, translation; requires controlled environment
\item Sensitive to expression changes
\item The most significant eigenfaces (basis vectors) encore more about lighting than about the faces
\item Head-on view
\end{itemize}
\end{itemize}



\end{document}
