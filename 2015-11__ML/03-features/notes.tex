\input ../notes-header.tex

\begin{document}

\notetitle{03}{Feature Extraction}

Categorical variables
\begin{itemize}
\item \textit{One of $K$ } or \textit{one-hot} encoding --- one binary feature per possible value
\item The importance of not encoding order where none exists
\item Text as explanatory variable $\implies$ encode as feature vectors
\item \fbox{Bag of words}
  \begin{itemize}
  \item Corpus (collection of documents)
  \item Vocabulary (set of unique words in document)
  \item Words = dimensions
  \item Order of words doesn't matter
  \item Order in vectors encodes words
  \item Binary: present or not
  \item \fbox{\tt CountVectorizer}, by default:
    \begin{itemize}
    \item converts to lowercase
    \item tokens
    \item stop words (mot vide) --- words in most documents don't convey much information
    \item stemming --- rule-based, drop suffixes\\
      (racinisation ou désuffixation : transformer des flexions en leur radical or racine)
    \item lemmatization -- find root form of word\\
      (lemmatisation : transformer en lemme (forme canonique))
    \end{itemize}
  \end{itemize}
\item \fbox{TF - IDF}
  \begin{itemize}
  \item 
  \end{itemize}


\item 
\end{itemize}

\end{document}
