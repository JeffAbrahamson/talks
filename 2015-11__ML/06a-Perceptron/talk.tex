\input ../talk-header.tex
\title
{ML Week}
\subtitle{0x06a \hspace{2mm}  Perceptron}

% If you wish to uncover everything in a step-wise fashion, uncomment
% the following command: 
%\beamerdefaultoverlayspecification{<+->}

\begin{document}

\begin{frame}
  \titlepage
\end{frame}


\begin{frame}
  \frametitle{Introducing Percepton}
  \begin{itemize}
  \item Supervised learning
  \item Binary classifier
  \item Linear classifier
  \item Permits online learning
  \end{itemize}
\end{frame}

\begin{frame}
  \frametitle{History}
  \begin{itemize}
  \item Invented in 1957 by Frank Rosenblatt
  \item Cornell University Aeronautical Laboratory
  \item Paid for by ONR
  \item First implementation in software (IBM 704)
  \item Intended to be a machine
  \item Designed for image recognition
  \end{itemize}
\end{frame}

\begin{frame}
  \frametitle{Hardware}
  \begin{itemize}
  \item Inputs = 400 CdS photocells
  \item Weights = potentiometers
  \item Tuning = electric motors
  \end{itemize}
\end{frame}

\begin{frame}
  \frametitle{Patch panel and potentiometers}
  \cimg{mark-1-perceptron.jpg}
\end{frame}

\begin{frame}
  \frametitle{Controversy}
  \begin{itemize}
  \item 1958, press conference, NYT
  \item Rosenblatt too optimistic
  \item 1969, Minsky and Papert
  \end{itemize}
\end{frame}

\begin{frame}
  \frametitle{Algorithm}

  \hspace{27mm}We want to learn a linear separator
  
  \begin{displaymath}
    f(x) = \left\{
    \begin{array}{ll}
      1 & \text{ if } w\cdot x + b > 0 \\[2mm]
      0 & \text{ otherwise}
    \end{array}\right.
  \end{displaymath}
  
  \hspace{27mm}where $w\in \mathbb{R}^{n+1}$.
\end{frame}

\begin{frame}
  \frametitle{Algorithm}
  \begin{enumerate}
  \item Initialise weights
  \item From input, compute output
  \item If correct, add $(\alpha \cdot\text{input})$ to weight
  \end{enumerate}
\end{frame}

\begin{frame}
  \frametitle{Convergence}
  \begin{itemize}
  \item If linearly separable, yes
  \item If not linearly separable, learning fails
  \end{itemize}
\end{frame}

\begin{frame}
  \phrase{feedforward neural network}
\end{frame}

%%%%%%%%%%%%%%%%%%%%%%%%%%%%%%%%%%%%%%%%%%%%%%%%%%%%%%%%%%%%%%%%%%%%%%
%\talksection{Break}

\begin{frame}
  \frametitle{Questions?}
  \centerline{\large\url{purple.com/talk-feedback}}
\end{frame}

\end{document}
