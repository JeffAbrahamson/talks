\documentclass{beamer}

\mode<presentation>
{
  \usetheme{default}
  \setbeamertemplate{navigation symbols}{}
  \setbeamertemplate{footline}[frame number]
  \setbeamertemplate{items}[circle]
  \usecolortheme{seahorse}
}

\usepackage[english]{babel}
\usepackage[utf8]{inputenc}
\usepackage{times}
\usepackage[T1]{fontenc}
\usepackage{url}
\usepackage[]{algorithm2e}
\usepackage{amsmath}
\usepackage{centernot}
\usepackage{xcolor}

\parskip=8 pt

\newcommand\topstrut{\rule{0pt}{2.6ex}}
\newcommand\bottomstrut{\rule[-1.2ex]{0pt}{0pt}}
\newcommand\doublestrut{\rule[-1.2ex]{0pt}{3.6ex}}

\newcommand\blue[1]{\textcolor{blue}{#1}}
\newcommand\red[1]{\textcolor{red}{#1}}
\newcommand\gray[1]{\textcolor{gray}{#1}}
\newcommand\purple[1]{\textcolor{purple}{#1}}
\newcommand\smallgray[1]{\textcolor{gray}{\footnotesize\it #1}}
\newcommand\prevwork[1]{\smallgray{#1}}
\newcommand\solo[1]{\centerline{#1}}
\newcommand\soloo[2]{\only<#1>{\solo{#2}}}
\newcommand\solopb[1]{\centerline{\parbox{.9\textwidth}{#1}}}
\newcommand\soloopb[2]{\only<#1>{\solopb{#2}}}
\newcommand\cimgf[1]{\centerline{\includegraphics[width=\textwidth]{#1}}}
\newcommand\cimg[1]{\centerline{\includegraphics[width=.9\textwidth]{#1}}}
\newcommand\cimgg[1]{\centerline{\includegraphics[width=.8\textwidth]{#1}}}
\newcommand\cimggg[1]{\centerline{\includegraphics[width=.7\textwidth]{#1}}}
\newcommand\cimgsm[1]{\centerline{\includegraphics[width=.4\textwidth]{#1}}}
\newcommand\cimgh[1]{\centerline{\includegraphics[height=.9\textwidth]{#1}}}

\title
{ZeroMQ}
%\subtitle{MLaaS}

\author[Abrahamson] {Jeff Abrahamson}\institute{Jellybooks}

\date[15 février 2016]

% Delete this, if you do not want the table of contents to pop up at
% the beginning of each subsection:
\AtBeginSubsection[]
{
  \begin{frame}<beamer>{Outline}
    \tableofcontents[currentsection,currentsubsection]
  \end{frame}
}

% If you wish to uncover everything in a step-wise fashion, uncomment
% the following command: 
%\beamerdefaultoverlayspecification{<+->}

\begin{document}

\begin{frame}
  \titlepage
\end{frame}

\begin{frame}
  \frametitle{C'est top !}

  ``0MQ is unbelievably cool---if you haven't got a project that needs it, \red{make one up}''

  Jon Gifford
\end{frame}

\begin{frame}
  \frametitle{How It Began}

  We took a normal TCP socket, injected it with a mix of radioactive
  isotopes stolen from a secret Soviet atomic research project,
  bombarded it with 1950-era cosmic rays, and put it into the hands of a
  drug-addled comic book author with a badly-disguised fetish for
  bulging muscles clad in spandex. Yes, ZeroMQ sockets are the
  world-saving superheroes of the networking bord.
\end{frame}

\begin{frame}
  \frametitle{Les problèmes}
  \center
  \only<1>{I/O, non-blocking}
  \only<2>{Composants dynamiques?

    \begin{itemize}
    \item client/serveur (stable?)
    \item serveur/serveur
    \item reconnection
    \end{itemize}
  }
  \only<3>{Wire format

    \begin{itemize}
    \item cadrage (framing)
    \item buffer overflow
    \item efficacité pour petit et grand message
    \end{itemize}
  }
  \only<4>{Quoi faire en attente?

    \begin{itemize}
    \item impossibilité de livraison
    \item attente de réponse
    \end{itemize}
  }
  \only<5>{Quoi faire avec nos files d'attente?

    Et s'il y a trop de messages?}
  \only<6>{Et les messages perdus?

        \begin{itemize}
        \item attendre
        \item redemander
        \item couche de fiabilité
        \end{itemize}
  }
  \only<7>{Abstraction de couche de transport

    \begin{itemize}
    \item inproc
    \item ipc
    \item tcp
    \item pgm
    \end{itemize}
  }
  \only<8>{Routing

    \begin{itemize}
    \item Envoie de messages à multiples clients
    \item réponse à qui?
    \end{itemize}
  }
  \only<9>{Génération d'API pour d'autres langages?}
  \only<10>{Changement d'architecture}
  \only<11>{Erreurs de réseau}

\end{frame}

\begin{frame}
  \frametitle{Un exemple vrai}

  \centerline{tyrannosaur}
  \centerline{rex.py}
\end{frame}

\begin{frame}
  \frametitle{Exemple}

  \centerline{hello world client/server}
\end{frame}

\begin{frame}
  \frametitle{Exemple}

    \centerline{hello world pub/sub (météo)}
\end{frame}

\begin{frame}
  \frametitle{Exemple}

  \centerline{task pipeline}

  \begin{itemize}
  \item répartition de charge
  \item fair queuing
  \end{itemize}
\end{frame}

\begin{frame}
  \frametitle{Exemple}

  \centerline{proxy}
\end{frame}

\begin{frame}
  \frametitle{Encryption}

  \centerline{CurveMQ}
\end{frame}

\begin{frame}
  \frametitle{Questions?}

  \centerline{\url{https://github.com/JeffAbrahamson/talks/}}
  \vspace{1cm}
  \centerline{\url{jeff@purple.com}}
\end{frame}

\end{document}
