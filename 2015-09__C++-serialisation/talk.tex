\documentclass[t]{beamer}

\mode<presentation>
{
  \usetheme{default}
  \setbeamertemplate{navigation symbols}{}
  \setbeamertemplate{footline}[frame number]
  \setbeamertemplate{items}[circle]
  \usecolortheme{seahorse}
}

\usepackage[english]{babel}
\usepackage[utf8]{inputenc}
\usepackage{times}
\usepackage[T1]{fontenc}
\usepackage{url}
\usepackage[]{algorithm2e}
\usepackage{amsmath}
\usepackage{centernot}
\usepackage{xcolor}
\usepackage{listings}
\lstset{ showstringspaces=false }

\parskip=8 pt

\newcommand\topstrut{\rule{0pt}{2.6ex}}
\newcommand\bottomstrut{\rule[-1.2ex]{0pt}{0pt}}
\newcommand\doublestrut{\rule[-1.2ex]{0pt}{3.6ex}}

\newcommand\blue[1]{\textcolor{blue}{#1}}
\newcommand\red[1]{\textcolor{red}{#1}}
\newcommand\gray[1]{\textcolor{gray}{#1}}
\newcommand\purple[1]{\textcolor{purple}{#1}}
\newcommand\smallgray[1]{\textcolor{gray}{\footnotesize\it #1}}
\newcommand\prevwork[1]{\smallgray{#1}}
\newcommand\solo[1]{\centerline{#1}}
\newcommand\soloo[2]{\only<#1>{\solo{#2}}}
\newcommand\solopb[1]{\centerline{\parbox{.9\textwidth}{#1}}}
\newcommand\soloopb[2]{\only<#1>{\solopb{#2}}}
\newcommand\cimgf[1]{\centerline{\includegraphics[width=\textwidth]{#1}}}
\newcommand\cimg[1]{\centerline{\includegraphics[width=.9\textwidth]{#1}}}
\newcommand\cimgg[1]{\centerline{\includegraphics[width=.8\textwidth]{#1}}}
\newcommand\cimggg[1]{\centerline{\includegraphics[width=.7\textwidth]{#1}}}
\newcommand\cimgsm[1]{\centerline{\includegraphics[width=.4\textwidth]{#1}}}
\newcommand\cimgh[1]{\centerline{\includegraphics[height=.9\textwidth]{#1}}}

\title
{Serialisation: a Comparison}
%\subtitle{MLaaS}

\author[Abrahamson] {Jeff Abrahamson}\institute{Jellybooks}

\date[21 September 2015]

% Delete this, if you do not want the table of contents to pop up at
% the beginning of each subsection:
%\AtBeginSubsection[]
%{
%  \begin{frame}<beamer>{Outline}
%    \tableofcontents[currentsection,currentsubsection]
%  \end{frame}
%}

% If you wish to uncover everything in a step-wise fashion, uncomment
% the following command: 
%\beamerdefaultoverlayspecification{<+->}

\begin{document}

\begin{frame}
  \titlepage
\end{frame}

\begin{frame}
  \frametitle{In pace requiescat}

  \vfill
  \solo{Corba}
\end{frame}

\begin{frame}
  \frametitle{Basics}

  \begin{itemize}
  \item Backwards compatible
  \item Forwards compatible
  \item Efficiency (space, time)
  \end{itemize}
\end{frame}

\begin{frame}
  \frametitle{Subtleties}

  \begin{itemize}
  \item Community
  \item Safety
  \item Accent
  \end{itemize}
\end{frame}

\begin{frame}
  \frametitle{Frameworks considered}

  Good:
  \begin{itemize}
  \item Thrift (Facebook, now Apache)
  \item Protobuf (Google)
  \end{itemize}

  Not so good:
  \begin{itemize}
  \item Boost Serialisation
  \item Json
  \end{itemize}

\end{frame}

\begin{frame}
  \frametitle{Don't Even}

  \begin{itemize}
  \item CORBA (1991) (cf. Vasa, 1628)
  \item XML-RPC (1998)
  \item SOAP (1998, successor to XML-RPC)
  \end{itemize}
\end{frame}

\begin{frame}
  \frametitle{RPC}

  \begin{itemize}
  \item Protobuf (no)
  \item Thrift (yes)
  \item Boost (no)
  \item Json (no)
  \item Cap'n Proto (yes)
  \end{itemize}
\end{frame}

\begin{frame}
  \frametitle{RPC}

  Some popular RPC communication libraries:
  \begin{itemize}
  \item ZMQ
  \item RabbitMQ
  \item Etch (?)
  \item ZeroC ICE (?)
  \item Apache Qpid and AMQP (?)
  \end{itemize}
  But this isn't a talk about RPC.
\end{frame}

\begin{frame}
  \frametitle{Protobuf and Thrift}

  \begin{itemize}
  \item IDL
  \item Proven scalable
  \item Community
  \item Similar efficiency
  \end{itemize}

\end{frame}

\begin{frame}
  \frametitle{Protobuf}
  \lstinputlisting[language=C++]{proto1.proto} 
\end{frame}

\begin{frame}
  \frametitle{Protobuf}

  \lstinputlisting[language=C++]{proto2.proto}
\end{frame}

\begin{frame}
  \frametitle{Protobuf}

  \lstinputlisting{proto3.cc}
\end{frame}

\begin{frame}
  \frametitle{Protobuf}

  \lstinputlisting[language=C++]{proto4.cc}
\end{frame}

\begin{frame}
  \frametitle{Protobuf}

  \lstinputlisting[language=C++]{proto5.cc}
\end{frame}

\begin{frame}
  \frametitle{Protobuf}

  \lstinputlisting[language=C++]{proto6.cc}
\end{frame}

\begin{frame}
  \frametitle{Protobuf}

  \lstinputlisting[language=C++]{proto7.cc}
\end{frame}

\begin{frame}
  \frametitle{Thrift}

  \lstinputlisting[language=C++]{thrift1.thrift}
\end{frame}

\begin{frame}
  \frametitle{Thrift}

  \lstinputlisting[language=C++]{thrift2.thrift}
\end{frame}

\begin{frame}
  \frametitle{Thrift}

  \lstinputlisting[language=C++]{thrift3.thrift}
\end{frame}

\begin{frame}
  \frametitle{Thrift}

  \lstinputlisting[language=C++]{thrift4.cc}
\end{frame}

\begin{frame}
  \frametitle{Thrift}

  \lstinputlisting[language=C++]{thrift5.cc}
\end{frame}

\begin{frame}
  \frametitle{Thrift}

  \lstinputlisting[language=C++]{thrift5a.cc}
\end{frame}

\begin{frame}
  \frametitle{Thrift}

  \lstinputlisting[language=C++]{thrift6.cc}
\end{frame}

\begin{frame}
  \frametitle{Thrift}

  \lstinputlisting[language=C++]{thrift7.cc}
\end{frame}

\begin{frame}
  \frametitle{Thrift}

  \lstinputlisting[language=C++]{thrift8.cc}
\end{frame}

\begin{frame}
  \frametitle{Thrift}

  \lstinputlisting[language=C++]{thrift9.cc}
\end{frame}

\begin{frame}
  \frametitle{Boost}

  \lstinputlisting[language=C++]{boost1.cc}
\end{frame}

\begin{frame}
  \frametitle{Boost}

  \lstinputlisting[language=C++]{boost2.cc}
\end{frame}

\begin{frame}
  \frametitle{Boost}

  \lstinputlisting[language=C++]{boost3.cc}
\end{frame}

\begin{frame}
  \frametitle{Boost}

  \lstinputlisting[language=C++]{boost4.cc}
\end{frame}

\begin{frame}
  \frametitle{Boost}

  \lstinputlisting[language=C++]{boost5.cc}
\end{frame}

\begin{frame}
  \frametitle{Boost}

  \lstinputlisting[language=C++]{boost6.cc}
\end{frame}

\begin{frame}
  \frametitle{Boost}

  \lstinputlisting[language=C++]{boost7.cc}
\end{frame}

\begin{frame}
  \frametitle{Boost}

  \lstinputlisting[language=C++]{boost8.cc}
\end{frame}

\begin{frame}
  \frametitle{Boost}

  \lstinputlisting[language=C++]{boost9.cc}
\end{frame}

\begin{frame}
  \frametitle{Boost}

  \lstinputlisting[language=C++]{boostA.cc}
\end{frame}

\begin{frame}
  \frametitle{Boost}

  \lstinputlisting[language=C++]{boostB.cc}
\end{frame}

\begin{frame}
  \frametitle{Boost}

  \lstinputlisting[language=C++]{boostC.cc}
\end{frame}

\begin{frame}
  \frametitle{Boost}

  \lstinputlisting[language=C++]{boostD.cc}
\end{frame}

\begin{frame}
  \frametitle{Json}

  \begin{itemize}
  \item Ubiquitous
  \item Human readable
  \item No validation or error checking
  \item Verbose
  \end{itemize}

  Cf. Bson, MessagePack, \dots
\end{frame}

\begin{frame}
  \frametitle{Questions?}

  %\centerline{Feedback: \url{http://purple.com/1}}
  %\vspace{1.5cm}
  \centerline{\url{https://github.com/JeffAbrahamson/talks/}}
  \vspace{1cm}
  \centerline{\url{jeff@purple.com}}
\end{frame}

\end{document}