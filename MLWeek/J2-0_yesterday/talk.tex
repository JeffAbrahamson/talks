\input ../talk-header.tex
\title
{ML Week}
\subtitle{Notes from yesterday}

% If you wish to uncover everything in a step-wise fashion, uncomment
% the following command: 
%\beamerdefaultoverlayspecification{<+->}

\begin{document}

\begin{frame}
  \titlepage
\end{frame}

\begin{frame}
  \frametitle{One vs Rest, One vs One}
  \only<1>{
    What I described yesterday:
    \begin{itemize}
    \item OvR (OvA): compute $k$ classifiers
    \item OvO: compute $k(k-1)/2$ classifiers
    \end{itemize}

    The missing point: the classifiers give scores, not just in/out answers.
  }
  \only<2>{
    One vs Rest:

    Accept the judgement of the classifier with the highest score.
  }
  \only<3>{
    One vs One:

    Classifiers vote.  Accept the class that gets the most votes.
  }
  \only<3>{
    Advantage:  Reduces multi-class classification to single-class classification.

    Disadvantage:  Classifier scores aren't necessarily comparable.  For example, classes
    may have very different numbers of members.
  }
\end{frame}

\begin{frame}
  \frametitle{Hyperparameters}
  \only<1>{
    \begin{itemize}
    \item The word hyperparameter is not well-defined.
    \item In most contexts, it is the parameters of the underlying distribution
    \item In training, we learn the parameters of the model
    \item We choose the hyperparameters to govern the training
    \item So we may want to experiment to learn the distribution
      parameters that best optimise our learned model's performance
    \end{itemize}
  }
\end{frame}



%%%%%%%%%%%%%%%%%%%%%%%%%%%%%%%%%%%%%%%%%%%%%%%%%%%%%%%%%%%%%%%%%%%%%%
%\talksection{Break}

\begin{frame}
  \frametitle{Questions?}
  \centerline{\large\url{purple.com/talk-feedback}}
\end{frame}

\end{document}
