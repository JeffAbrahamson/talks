\input talk-header.tex
\usepackage{centernot}
\usepackage[]{algorithm2e}

\title
{Mais qu'est-ce que je fous?}
\subtitle{comment glander et apprendre les maths en même temps}

\begin{document}

\begin{frame}
  \titlepage
\end{frame}

%%%%%%%%%%%%%%%%%%%%%%%%%%%%%%%%%%%%%%%%%%%%%%%%%%%%%%%%%%%%%%%%%%%%%%
%\talksection{Break}

\begin{frame}
  \cimg{mass.png}
\end{frame}

\begin{frame}
  \phrase{Le problème}
\end{frame}

\begin{frame}
  \frametitle{Le code}
  Avant j'utilisais ratpoison :
  \vspace{1cm}
  \lstinputlisting[language=bash, firstline=23, lastline=24]{ratpoison-gtd}
\end{frame}

\begin{frame}
  \frametitle{Le code}
  Maintenant j'utilise i3 :
  \vspace{1cm}
  \lstinputlisting[language=bash, firstline=38, lastline=40]{i3-gtd}
\end{frame}

\begin{frame}[fragile]
  \frametitle{Les données}
  Et ça donne :
  \vspace{1cm}
  \begin{Verbatim}[fontsize=\tiny]
1458399349 emacs@starshine - talk-header.tex : /home/jeff/src/jma/talks/2016-03__breizhcamp-que-je-fous/talk-header.tex
1458399409 emacs@starshine - talk.tex : /home/jeff/src/jma/talks/2016-03__breizhcamp-que-je-fous/talk.tex
1458399469 listings - How to put C++ source code into beamer slides - TeX - LaTeX Stack Exchange - Mozilla Firefox
1458399529 emacs@starshine - talk.tex : /home/jeff/src/jma/talks/2016-03__breizhcamp-que-je-fous/talk.tex
1458399589 talk.pdf — Mais qu'est-ce que je fous? - comment glander et apprendre les maths en même temps
1458399649 emacs@starshine - talk.tex : /home/jeff/src/jma/talks/2016-03__breizhcamp-que-je-fous/talk.tex
1458399709 emacs@starshine - talk.tex : /home/jeff/src/jma/talks/2016-03__breizhcamp-que-je-fous/talk.tex
1458399769 emacs@starshine - macros.tex : /home/jeff/src/jma/talks/2016-03__breizhcamp-que-je-fous/macros.tex
1458399829 emacs@starshine - talk.tex : /home/jeff/src/jma/talks/2016-03__breizhcamp-que-je-fous/talk.tex
1458399890 emacs@starshine - talk.tex : /home/jeff/src/jma/talks/2016-03__breizhcamp-que-je-fous/talk.tex
1458399950 talk.pdf — Mais qu'est-ce que je fous? - comment glander et apprendre les maths en même temps
1458400010 jeff@starshine:~/src/jma/talks/2016-03__breizhcamp-que-je-fous
\end{Verbatim}
\end{frame}

\begin{frame}
  \phrase{Quelles questions puis-je poser?}
\end{frame}

\begin{frame}
  \phrase{Machine Learning}
\end{frame}

\begin{frame}
  \phrase{Data Science}
\end{frame}

\begin{frame}
  \phrase{Statistics}
\end{frame}

\begin{frame}
  \cimgggg{boxplot-vs-pdf.png}
\end{frame}

\begin{frame}
  \cimg{maths.jpg}
\end{frame}

\begin{frame}
  \frametitle{Maths}
  \phrase{Vector Space}
  \vspace{7mm}
  \phrase{Features}
  \vspace{7mm}
  \phrase{Feature engineering}
\end{frame}

\begin{frame}
  \cimg{gtd-history.png}
\end{frame}

\begin{frame}
  \cimg{gtd-history-by-host.png}
\end{frame}

\begin{frame}
  \cimg{gtd-recent-days.png}
  % Note that does not include tablet use.
  % And talk about discovering long tea breaks
\end{frame}

\begin{frame}
  \cimg{gtd-pauses-30.png}
  % Note that I'm not measuring what I do on a tablets
\end{frame}

\begin{frame}
  \cimg{gtd-pauses-60.png}
  % Note that I'm not measuring what I do on a tablets
\end{frame}

\begin{frame}
  \cimg{gtd-pauses-90.png}
  % Note that I'm not measuring what I do on a tablets
\end{frame}

\begin{frame}
  \cimg{gtd-pauses-120.png}
  % Note that I'm not measuring what I do on a tablets
\end{frame}

\begin{frame}
  \frametitle{Bag of Words}
  \phrase{sac de mots}
\end{frame}

\begin{frame}[t]
  \frametitle{Bag of Words}
  \vspace{1cm}
  Exemple :

  \begin{quote}
    Il est nuit. La cabane est pauvre, mais bien close.\\
    Le logis est plein d'ombre et l'on sent quelque chose\\
    Qui rayonne à travers ce crépuscule obscur.\\
    Des filets de pêcheur sont accrochés au mur.\\
    Au fond, dans l'encoignure où quelque humble vaisselle\\
    Aux planches d'un bahut vaguement étincelle,\\
    On distingue un grand lit aux longs rideaux tombants.\\
    Tout près, un matelas s'étend sur de vieux bancs,\\
    Et cinq petits enfants, nid d'âmes, y sommeillent\\
    La haute cheminée où quelques flammes veillent\\
    Rougit le plafond sombre, et, le front sur le lit,\\
    Une femme à genoux prie, et songe, et pâlit.\\
    C'est la mère. Elle est seule. Et dehors, blanc d'écume,\\
    Au ciel, aux vents, aux rocs, à la nuit, à la brume,\\
    Le sinistre océan jette son noir sanglot.
  \end{quote}
\end{frame}

\begin{frame}[t]
  \frametitle{Bag of Words}
  \vspace{1cm}
  Exemple (plus simple) :

  \begin{quote}
    Il est nuit. La cabane est pauvre, mais bien close.\\
    Le logis est plein d'ombre et l'on sent quelque chose\\
  \end{quote}
\end{frame}

\begin{frame}[t]
  \frametitle{Bag of Words}
  \vspace{1cm}
  Exemple (plus simple) :

  \begin{quote}
    Il est nuit. La cabane est pauvre, mais bien close.\\
    Le logis est plein d'ombre et l'on sent quelque chose\\
  \end{quote}

  \only<2>{
    % vectorizer = CountVectorizer(analyzer='word')
    % ft = vectorizer.fit_transform(pauvres_gens)
    % # vectorizer.inverse_transform(ft)
    % ft.todense()

    \vspace{1cm}
    \hspace{2cm}\texttt{[1, 1, 0, 1, 2, 0, 1, 1, 0, 0, 1, 1, 0, 0, 1, 0, 0, 0]}\\[2mm]
    \hspace{2cm}\texttt{[0, 0, 1, 0, 1, 1, 0, 0, 1, 1, 0, 0, 1, 1, 0, 1, 1, 1]}
  }
\end{frame}

\begin{frame}
  Phrase from corpus with top 10 matches
\end{frame}

\begin{frame}
  Same phrase from corpus with top 90--100 matches
\end{frame}

\begin{frame}
  Same phrase using TfIdf
\end{frame}

\begin{frame}
  Same phrase using TfIdf, showing 90--100 matches
\end{frame}

\begin{frame}
  \frametitle{TF - IDF}
  \blue{
    \begin{displaymath}
      TF_{td} = \frac{f_{td}}{\max_k f_{kd}} \qquad\qquad
      IDF_t = \log_2\left( \frac{N}{n_t} \right)
    \end{displaymath}
  }
  \blue{
    \begin{displaymath}
      TF\mbox{-}IDF_{td} = TF_{td} \cdot IDF_t
    \end{displaymath}
  }

  with
  \begin{align*}
    f_{td} &= \mbox{frequency of word (term) $t$ in document $d$} \\
    N &= \mbox{number of documents}\\
    n_t &= \mbox{number of documents containing term $t$}
  \end{align*}
\end{frame}

\begin{frame}
  \phrase{$k$-means}
\end{frame}

\begin{frame}
  \cimgggg{cluster-1.png}
\end{frame}

\begin{frame}
  \cimg{cluster-2.png}
\end{frame}

\begin{frame}
  \cimg{cluster-3.png}
\end{frame}

\begin{frame}
  \cimgg{cluster-4.png}
\end{frame}

\begin{frame}
  Visualize date-time with clusters
\end{frame}

\begin{frame}
  $n$-grams
\end{frame}

\begin{frame}
  Integer programming
\end{frame}



% ======================================================================
% End

\begin{frame}
  \frametitle{Resources}
  \cimggg{nmlm.png}

  \vspace{5mm}
  \centerline{\url{http://www.meetup.com/Nantes-Machine-Learning-Meetup/}}
\end{frame}

\begin{frame}
  \frametitle{Resources}
  \cimggg{ml-week.png}

  \vspace{5mm}
  \centerline{\url{http://www.ml-week.com/}}
\end{frame}

\begin{frame}
  \phrase{Questions?}
\end{frame}

\end{document}
