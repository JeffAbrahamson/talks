\documentclass{beamer}

\mode<presentation>
{
  \usetheme{default}
  \setbeamertemplate{navigation symbols}{}
  \setbeamertemplate{footline}[frame number]
  \setbeamertemplate{items}[circle]
  \usecolortheme{seahorse}
}

\usepackage[english]{babel}
\usepackage[utf8]{inputenc}
\usepackage{times}
\usepackage[T1]{fontenc}
\usepackage{url}
\usepackage[]{algorithm2e}
\usepackage{amsmath}
\usepackage{centernot}
\usepackage{xcolor}

\parskip=8 pt

\newcommand\topstrut{\rule{0pt}{2.6ex}}
\newcommand\bottomstrut{\rule[-1.2ex]{0pt}{0pt}}
\newcommand\doublestrut{\rule[-1.2ex]{0pt}{3.6ex}}

\newcommand\blue[1]{\textcolor{blue}{#1}}
\newcommand\red[1]{\textcolor{red}{#1}}
\newcommand\gray[1]{\textcolor{gray}{#1}}
\newcommand\purple[1]{\textcolor{purple}{#1}}
\newcommand\smallgray[1]{\textcolor{gray}{\footnotesize\it #1}}
\newcommand\prevwork[1]{\smallgray{#1}}
\newcommand\solo[1]{\centerline{#1}}
\newcommand\soloo[2]{\only<#1>{\solo{#2}}}
\newcommand\solopb[1]{\centerline{\parbox{.9\textwidth}{#1}}}
\newcommand\soloopb[2]{\only<#1>{\solopb{#2}}}
\newcommand\cimgf[1]{\centerline{\includegraphics[width=\textwidth]{#1}}}
\newcommand\cimg[1]{\centerline{\includegraphics[width=.9\textwidth]{#1}}}
\newcommand\cimgg[1]{\centerline{\includegraphics[width=.8\textwidth]{#1}}}
\newcommand\cimggg[1]{\centerline{\includegraphics[width=.7\textwidth]{#1}}}
\newcommand\cimgsm[1]{\centerline{\includegraphics[width=.4\textwidth]{#1}}}
\newcommand\cimgh[1]{\centerline{\includegraphics[height=.9\textwidth]{#1}}}

\title
{Predicting the Future}
\subtitle{MLaaS}

\author[Abrahamson] {Jeff Abrahamson}\institute{Jellybooks}

\date[6 juin 2015]

% Delete this, if you do not want the table of contents to pop up at
% the beginning of each subsection:
\AtBeginSubsection[]
{
  \begin{frame}<beamer>{Outline}
    \tableofcontents[currentsection,currentsubsection]
  \end{frame}
}

% If you wish to uncover everything in a step-wise fashion, uncomment
% the following command: 
%\beamerdefaultoverlayspecification{<+->}

\begin{document}

\begin{frame}
  \vfill
  \cimggg{web2day.png}
  
  \vfill
  \centerline{\fcolorbox{purple}{pink}{\blue{ici : Predicting the Future}}}
  \vfil
\end{frame}

\begin{frame}
  \titlepage
\end{frame}

\begin{frame}
  \frametitle{}

  \solopb{Prediction is very difficult, especially about the future.}

  \vspace{1cm}
  {\small\it Niels Bohr (1970) (or Yogi Bera or Piet Hein, or Steincke
    (1948)), but probably Markus M. Ronner (1918)}
\end{frame}

\begin{frame}
  \frametitle{What's a simpler problem?}

  \vfill
  \only<1>{\cimgsm{k-pts.png}}
  \only<2>{\cimgsm{k-pts-line.png}}
  \only<3>{\cimgsm{k-pts-line-plus.png}}
  \only<4>{\cimgsm{k-pts-line-error.png}}

  \note{
    Or maybe these are classes: dogs vs cats.

    Problem: high dimension.
    \begin{itemize}
    \item Thinking of higher dimensions
    \item Strange things happen
    \end{itemize}

  }
\end{frame}

\begin{frame}
  \frametitle{Training a Model}

  \solopb{Known points, draw a line.\\ \blue{Model: the line}}
  \note{There exist more complex models.}

\end{frame}

\begin{frame}
  \frametitle{Predict}

  \solopb{New point, \blue{which side of line?}}

  \note {
    Time series possible, more complex than for today.

    Maybe add confidence.
  }

\end{frame}

\begin{frame}[t]
  \frametitle{Evaluation}

  \vspace{1cm}
  Example :
  \begin{itemize}
  \item Rare illness, .001\% of population each year
  \item Treatable early, but expensive and hurts
  \item Or\dots dead in a month
  \end{itemize}

  \only<2>{
    \blue{Training: past cases (know outcomes)}
  }
  \only<3>{
    \purple{\tt \# First pass, reliability = 99.999\% }\\
    \blue{\tt def has\_disease(person): return false}
    \note{Best to look at false positives and false negatives.}
  }
\end{frame}

\begin{frame}
  \frametitle{Examples}

  \begin{itemize}
  \item Credit scoring
  \item Fraud detection
  \item Personalization
  \item Spam on website
  \item Recommend a book, a movie, a holiday
  \end{itemize}

\end{frame}

\begin{frame}
  \frametitle{Services}

  \begin{itemize}
  \item Amazon Machine Learning (free: 12 months)
  \item Google Prediction API $\subset$ Cloud Platform (free: 6 months)
  \item PredicSis
  \item BigML (free: in development mode)\\[3mm]
  \item Apache Mahout (free: beer and speech)
  \item Scikit-learn (free: beer and speech, pipeline sold separately)
  \end{itemize}
\end{frame}

\begin{frame}
  \frametitle{Options}

  \vfil
  
  \begin{itemize}
  \item Batch Prediction API
  \item Real-Time API
  \end{itemize}

  \vspace{1cm}
  \prevwork{Amazon-speak}

\end{frame}

\begin{frame}
  \frametitle{Comparison}

  \begin{itemize}
  \item BigML and PredicSis faster
  \item Amazon and PredicSis slightly more accurate
  \item YMMV
  \end{itemize}

  \vfill
  \prevwork{\url{http://www.louisdorard.com/blog/machine-learning-apis-comparison}}

\end{frame}

\begin{frame}
  \frametitle{Is it Magic?}

  \solo{\cimg{log-reg.png}}

\end{frame}

\begin{frame}[t]
  \frametitle{Is it Magic?}

  \vspace{1cm}
  \begin{itemize}
  \item Clean data
  \item Feature engineering
  \item (Sometimes) Choose a model
  \item Train model
  \item Send queries, receive predictions
  \end{itemize}

  \only<2>{Free pipeline (except scikit-learn)}

\end{frame}

\begin{frame}
  \frametitle{Is it Real?}

  \soloo{1}{\cimggg{ads-save.png}}
  \soloo{2}{\cimggg{ads-prostate.png}}
  \soloo{3}{\cimg{ads-amazon.png}}

\end{frame}

\begin{frame}
  \frametitle{Free as in Beer}

  \solo{\cimgsm{beer.png}}
  
\end{frame}

\begin{frame}
  \frametitle{Free as in Speech}

  \solo{\cimg{freedom.png}}
  
\end{frame}

\begin{frame}
  \frametitle{Reflections}

  \begin{itemize}
  \item Data is most of the work: clean, select features, understand
  \item Choosing models requires knowledge
  \item Big data sets aren't always big
  \item If you need performance, \textit{maybe}
  \end{itemize}

\end{frame}

\begin{frame}
  \frametitle{More Reading}

  \prevwork{\url{http://www.programmableweb.com/news/6-questions-you-should-ask-about-prediction-apis/analysis/2014/09/12}}
  
\end{frame}

\begin{frame}[t]
  \frametitle{Sounds Easy?}

  \vspace{1cm}
  \cimg{nmlm.png}

  \prevwork{\centerline{\url{http://www.meetup.com/Nantes-Machine-Learning-Meetup/}}}

  \only<2>{\vspace{1cm} And there are jobs.}
\end{frame}

\begin{frame}
  \frametitle{Sounds hard?}

  \cimggg{ml-week.png}
  \vspace{5mm}
  \centerline{2--6 novembre 2015}

  \prevwork{\centerline{\url{http://ml-week.com/}}}
\end{frame}

\begin{frame}
  \frametitle{Questions?}

  \centerline{Feedback: \url{http://purple.com/1}}
  \vspace{1.5cm}
  \centerline{\url{https://github.com/JeffAbrahamson/talks/}}
  \vspace{1cm}
  \centerline{\url{jeff@purple.com}}
\end{frame}

%   \prevwork{\url{http://www.louisdorard.com/blog/machine-learning-apis-comparison}}
%   \prevwork{\url{https://github.com/louisdorard/papiseval}}
%   \prevwork{\url{http://www.louisdorard.com/about/}}
%   \prevwork{\url{http://www.rudebaguette.com/2014/06/16/connected-objects-predicting-next-move/}}
%   \prevwork{\url{http://www.louisdorard.com/blog/building-a-business-around-machine-learning-apis}}
%   \prevwork{\url{http://www.louisdorard.com/blog/predict-abstention-rates-open-data}}
%   \prevwork{\url{http://www.louisdorard.com/guest/everyone-can-do-data-science}}
%   \prevwork{\url{http://www.programmableweb.com/news/6-questions-you-should-ask-about-prediction-apis/analysis/2014/09/12}}

\end{document}