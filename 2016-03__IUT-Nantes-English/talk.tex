\input talk-header.tex
\usepackage{centernot}
\usepackage[]{algorithm2e}

\title
{Machine Learning}
\subtitle{About those machines that are going to eat your job}

\begin{document}

\begin{frame}
  \titlepage
\end{frame}



%%%%%%%%%%%%%%%%%%%%%%%%%%%%%%%%%%%%%%%%%%%%%%%%%%%%%%%%%%%%%%%%%%%%%%
%\talksection{Break}

\begin{frame}
  % https://www.pexels.com/photo/adult-beautiful-celebration-cozy-332090/
  % https://static.pexels.com/photos/332090/pexels-photo-332090.jpeg
  \cimgw{diner.jpg}
\end{frame}

\begin{frame}
  \phrase{Machine Learning}
\end{frame}

\begin{frame}
  \phrase{Data Science}
\end{frame}

\begin{frame}
  \phrase{Statistics}
\end{frame}

\begin{frame}
  \cimghh{boxplot-vs-pdf.png}
\end{frame}

\begin{frame}
  \cimg{maths.jpg}
\end{frame}

\begin{frame}
  \frametitle{Maths}
  \phrase{Vector Space}
  \vspace{7mm}
  \phrase{Features}
  \vspace{7mm}
  \phrase{Feature engineering}
\end{frame}

\begin{frame}
  \vspace{4mm}
  
  \cimg{titanic-titles.png}
  \vspace{-3mm}
  \prevwork{Kaggle}
\end{frame}

\begin{frame}
  \cimg{reading-time-1.png}
  \prevwork{Jellybooks}
\end{frame}

\begin{frame}
  \cimg{reading-time-2.png}
  \prevwork{Jellybooks}
\end{frame}

\begin{frame}
  \cimggg{regression-line-1.png}
\end{frame}

\begin{frame}
  \cimggg{regression.png}
\end{frame}

\begin{frame}
  \cimg{logreg.png}
\end{frame}

\begin{frame}
  \cimgg{separator-1.png}
\end{frame}

\begin{frame}
  \cimgg{separator-2.png}
\end{frame}

\begin{frame}
  \cimgg{separator-3.png}
\end{frame}

\begin{frame}
  \cimggg{cluster-1.png}
\end{frame}

\begin{frame}
  \cimg{cluster-2.png}
\end{frame}

\begin{frame}
  \cimg{cluster-3.png}
\end{frame}

\begin{frame}
  \cimg{cluster-4.png}
\end{frame}

\begin{frame}
  \cimg{brisk-image-descriptors.png}
  \prevwork{Eddie Bell @ Lyst}
\end{frame}

\begin{frame}
  \cimg{results-1.png}
  \prevwork{Eddie Bell @ Lyst}
\end{frame}

\begin{frame}
  \cimg{results-2.png}
  \prevwork{Eddie Bell @ Lyst}
\end{frame}

\begin{frame}
  \cimg{results-3.png}
  \prevwork{Eddie Bell @ Lyst}
\end{frame}

\begin{frame}
  \cimg{image-filter.png}
\end{frame}

\begin{frame}
  \cimg{word2vec-1.png}
\end{frame}

\begin{frame}
  \cimg{word2vec-2.png}
  \prevwork{Eddie Bell @ Lyst}
\end{frame}

\begin{frame}
  \cimg{brain-1.png}
  \prevwork{Google?}
\end{frame}

\begin{frame}
  \cimg{brain-2.png}
  \prevwork{Google?}
\end{frame}

\begin{frame}
  \cimg{shakespeare.png}
  \prevwork{\url{https://flowingdata.com/2015/12/30/shakespeare-tragedies-as-network-graphs/}}
  % Here we're looking at the structure of Shakespeare's tragedies
  % through character interactions.  Nodes are characters, edges
  % indicate they appear in the same scene.
\end{frame}

\begin{frame}
  \cimghh{merchant-of-venice.jpg}
\end{frame}

\begin{frame}
  \cimg{shakespeare-summarised.png}
  \prevwork{\url{http://www.nand.io/visualisation/understanding-shakespeare}}
  % Shakespeare’s plays as summar­ised by a soft­ware algorithm that
  % chooses the most repres­ent­ative sentence (or part of text) from
  % a speech as determ­ined by the frequency of the words it
  % contains. Thus, the algorithm chooses the one sentence that
  % contains most of the words that occur within the entire speech.
\end{frame}

\begin{frame}
  \cimghh{fourier-1.png}
  \prevwork{\url{http://www.toptal.com/algorithms/shazam-it-music-processing-fingerprinting-and-recognition}}
\end{frame}

\begin{frame}
  \cimg{fourier-2.png}
  \prevwork{\url{https://www.ee.columbia.edu/~dpwe/papers/Wang03-shazam.pdf}}
\end{frame}

\begin{frame}
  \cimggg{fourier-hash.png}
  \prevwork{\url{https://www.ee.columbia.edu/~dpwe/papers/Wang03-shazam.pdf}}
\end{frame}

\begin{frame}
  \cimg{google-dc.png}
  \prevwork{Google}
\end{frame}

\begin{frame}
  \vfill
  \phrase{Big data?}

  \vfill
  \prevwork{\url{http://aadrake.com/command-line-tools-can-be-235x-faster-than-your-hadoop-cluster.html}}
\end{frame}

\begin{frame}
  \frametitle{Some maths\dots}

  \begin{itemize}
  \item 100 features
  \item int \textit{(4 bytes per feature)}
  \item 16 GB RAM
  \item $O(n\log n)$ in RAM
  \end{itemize}

  \begin{displaymath}
    400 N + (400 N)\log (400 N)
    < 
    16 \cdot 10^{9}
  \end{displaymath}

  \only<2>{$$N \approx 2\cdot 10^6$$}
\end{frame}


% ======================================================================
% End

\begin{frame}
  % https://www.flickr.com/photos/romainguy/276314973/
  % http//c1.staticflickr.com/1/100/276314973_e68bc8f89e_b.jpg
  % CC0 license
  \cimgwb{warriors.jpg}

  \vspace{-3.1cm}
  \phrase{\Huge\orange{Questions?}\hspace{5cm}}
  \vspace{2.1cm}

\end{frame}

\end{document}
