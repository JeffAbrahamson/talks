\documentclass[t]{beamer}

\mode<presentation>
{
  \usetheme{default}
  \setbeamertemplate{navigation symbols}{}
  \setbeamertemplate{footline}[frame number]
  \setbeamertemplate{items}[circle]
  \usecolortheme{seahorse}
}

\usepackage[english]{babel}
\usepackage[utf8]{inputenc}
\usepackage{times}
\usepackage[T1]{fontenc}
\usepackage{url}
\usepackage{amsmath}

\parskip=8 pt

\newcommand\topstrut{\rule{0pt}{2.6ex}}
\newcommand\bottomstrut{\rule[-1.2ex]{0pt}{0pt}}
\newcommand\doublestrut{\rule[-1.2ex]{0pt}{3.6ex}}

\newcommand\blue[1]{\textcolor{blue}{#1}}
\newcommand\red[1]{\textcolor{red}{#1}}
\newcommand\gray[1]{\textcolor{gray}{#1}}
\newcommand\smallgray[1]{\textcolor{gray}{\small\it #1}}
\newcommand\prevwork[1]{\smallgray{#1}}
\newcommand\cimg[1]{\centerline{\includegraphics[width=.9\textwidth]{#1}}}
\newcommand\cimgg[1]{\centerline{\includegraphics[width=.8\textwidth]{#1}}}
\newcommand\cimggg[1]{\centerline{\includegraphics[width=.7\textwidth]{#1}}}
\newcommand\cimgsm[1]{\centerline{\includegraphics[width=.4\textwidth]{#1}}}

\newcommand\talksection[1]{\section{#1}
\begin{frame}
  \vfill\Huge\bf\blue{\centerline{#1}}
\end{frame}
}

% The differential in an integral.
% After a function or a fraction, the \, may not be desired, see \DD.
% It is as much art, taste, and consistency as norms and science.
\newcommand\D[1]{\,\mathrm{d}{#1}}
\newcommand\DD[1]{\mathrm{d}{#1}}
\newcommand\E[0]{\mathbf{E}}
\newcommand\var[0]{\mathbf{Var}}
\newcommand\N[0]{\mathcal{N}}
\newcommand\R[0]{\mathbb{R}}

\title
{Statistics for Machine Learning and Big Data}
\subtitle{An Introduction\\[6mm] Part 3: regression}

\author[Abrahamson] {Jeff Abrahamson}

% If you wish to uncover everything in a step-wise fashion, uncomment
% the following command: 
%\beamerdefaultoverlayspecification{<+->}

\begin{document}

\begin{frame}
  \titlepage
\end{frame}

%%%%%%%%%%%%%%%%%%%%%%%%%%%%%%%%%%%%%%%%%%%%%%%%%%%%%%%%%%%%%%%%%%%%%%
\talksection{Regression}

\begin{frame}
  \frametitle{Linear models}

  \textbf{Problem:}  We have a set of points $\{(x_i, y_i)\}$.  Given a new $x$
  value, we'd like to predict $y$.

  \textbf{Linear model:}  We'll assume there exists a linear relationship
  $y=\beta_0 + \beta_1 x$ that offers a good approximation to the data.

  \only<2>{
    We call $x$ the \textbf{explanatory} or \textbf{predictor}
    variable.

    We call $y$ the \textbf{response} variable.
  }

  \only<3>{Example:
    \cimgsm{regression-line-1.png}
  }

  \only<4>{Example:
    \cimgsm{regression-line-2.png}
  }

  \only<5>{Example:
    \cimgsm{regression-line-3.png}
  }

  \only<6>{Example:
    \cimg{regression-line-4.png}
  }

  \note{
    In the real world, there's nearly always noise.

    Sometimes there are other lesser effects as well.
  }

\end{frame}

\begin{frame}
  \frametitle{}

  \note{

  }

\end{frame}

\begin{frame}
  \frametitle{}

  \note{

  }

\end{frame}

\begin{frame}
  \frametitle{}

  \note{

  }

\end{frame}

\begin{frame}
  \frametitle{}

  \note{

  }

\end{frame}

\begin{frame}
  \frametitle{}

  \note{

  }

\end{frame}

\begin{frame}
  \frametitle{}

  \note{

  }

\end{frame}

\begin{frame}
  \frametitle{}

  \note{

  }

\end{frame}

\begin{frame}
  \frametitle{}

  \note{

  }

\end{frame}

\begin{frame}
  \frametitle{}

  \note{

  }

\end{frame}

\begin{frame}
  \frametitle{}

  \note{

  }

\end{frame}

\begin{frame}
  \frametitle{}

  \note{

  }

\end{frame}

\begin{frame}
  \frametitle{}

  \note{

  }

\end{frame}

\begin{frame}
  \frametitle{}

  \note{

  }

\end{frame}

\begin{frame}
  \frametitle{}

  \note{

  }

\end{frame}

\begin{frame}
  \frametitle{}

  \note{

  }

\end{frame}

\begin{frame}
  \frametitle{}

  \note{

  }

\end{frame}

\begin{frame}
  \frametitle{}

  \note{

  }

\end{frame}

\begin{frame}
  \frametitle{}

  \note{

  }

\end{frame}

\begin{frame}
  \frametitle{}

  \note{

  }

\end{frame}

\begin{frame}
  \frametitle{}

  \note{

  }

\end{frame}

\begin{frame}
  \frametitle{}

  \note{

  }

\end{frame}

\begin{frame}
  \frametitle{Questions?}
  \vspace{3cm}
  \centerline{\large\url{purple.com/talk-feedback}}
\end{frame}

\talksection{Break}

\end{document}
