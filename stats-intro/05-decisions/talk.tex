\documentclass[t]{beamer}

\input ../setup.tex

\title
{Statistics for Machine Learning and Big Data}
\subtitle{An Introduction\\[6mm] Part 5: Decision algorithms}

\author[Abrahamson] {Jeff Abrahamson}

% If you wish to uncover everything in a step-wise fashion, uncomment
% the following command: 
%\beamerdefaultoverlayspecification{<+->}

\begin{document}

\begin{frame}
  \titlepage
\end{frame}


\begin{frame}
  \frametitle{$k$-means}

  \cimg{k-means.jpg}

  \prevwork{\url{mathworks.com}}
  
  \note{

    Note that this is not related to kNN.

    Explain:
    \begin{itemize}
    \item Choose $k$
    \item Choose $k$ centroids
    \item Compute distances and assign each point to closest centroid
    \item Now re-compute centroids based on classes
    \item Now re-assign based on closest
    \item Converges because distances get smaller
    \end{itemize}

    Parameter sweep is a possibility for choosing $k$.

    Assumes roughly spherical clusters.

    Assignment:
    \begin{itemize}
    \item Pick $k$ random centroids from sample (\textit{Forgy} method).
    \item Assign to $k$ random clusters (\textit{random partitions} method).
    \end{itemize}

    Uses
    \begin{itemize}
    \item Vector quantization (picking hopefully prototypical samples)
    \end{itemize}

  }
\end{frame}

\begin{frame}
  \frametitle{Examples}

  \vfill
  \centerline{\huge{\bf Code time}}
\end{frame}

\begin{frame}
  \frametitle{Discussion}

  When should we use $k$-means vs logistic regression?
  
\end{frame}

%%%%%%%%%%%%%%%%%%%%%%%%%%%%%%%%%%%%%%%%%%%%%%%%%%%%%%%%%%%%%%%%%%%%%%
\talksection{\dots}

\begin{frame}
  \frametitle{}


  \note{

  }
\end{frame}


%%%%%%%%%%%%%%%%%%%%%%%%%%%%%%%%%%%%%%%%%%%%%%%%%%%%%%%%%%%%%%%%%%%%%%
\talksection{Break}

\begin{frame}
  \frametitle{Questions?}
  \vspace{3cm}
  \centerline{\large\url{purple.com/talk-feedback}}
\end{frame}

\end{document}
