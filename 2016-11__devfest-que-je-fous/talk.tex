\input talk-header.tex
\usepackage{centernot}
\usepackage[]{algorithm2e}

\title
{Mais qu'est-ce que je fous?}
\subtitle{comment glander et apprendre les maths en même temps}

\begin{document}
\huge

\begin{frame}
  \titlepage
\end{frame}

%%%%%%%%%%%%%%%%%%%%%%%%%%%%%%%%%%%%%%%%%%%%%%%%%%%%%%%%%%%%%%%%%%%%%%
%\talksection{Break}

\begin{frame}
  \frametitle{I will lie to you}
  \phrase{train vs plane}
\end{frame}

\begin{frame}
  \phrase{Le problème}
\end{frame}

\begin{frame}
  \frametitle{Le code}
  Avant j'utilisais ratpoison :
  \vspace{1cm}
  \lstinputlisting[language=bash, firstline=23, lastline=24]{ratpoison-gtd}
\end{frame}

\begin{frame}
  \frametitle{Le code}
  Maintenant j'utilise i3 :
  \vspace{1cm}
  \lstinputlisting[language=bash, firstline=38, lastline=40]{i3-gtd}
\end{frame}

\begin{frame}[fragile]
  \frametitle{Les données}
  Et ça donne :
  \vspace{1cm}
  \begin{Verbatim}[fontsize=\tiny]
1478605245 emacs@birdsong - talk.tex : /home/jeff/src/jma/talks/2016-11__devfest-que-je-fous/talk.tex
1478605305 emacs@birdsong - talk.tex : /home/jeff/src/jma/talks/2016-11__devfest-que-je-fous/talk.tex
1478605365 emacs@birdsong - talk.tex : /home/jeff/src/jma/talks/2016-11__devfest-que-je-fous/talk.tex
1478605425 talk.pdf — Mais qu'est-ce que je fous? - comment glander et apprendre les maths en même temps
1478605486 emacs@birdsong - talk.tex : /home/jeff/src/jma/talks/2016-11__devfest-que-je-fous/talk.tex
1478605546 emacs@birdsong - talk.tex : /home/jeff/src/jma/talks/2016-11__devfest-que-je-fous/talk.tex
1478605606 emacs@birdsong - talk.tex : /home/jeff/src/jma/talks/2016-11__devfest-que-je-fous/talk.tex
1478605666 emacs@birdsong - talk.tex : /home/jeff/src/jma/talks/2016-11__devfest-que-je-fous/talk.tex
1478605726 emacs@birdsong - talk.tex : /home/jeff/src/jma/talks/2016-11__devfest-que-je-fous/talk.tex
1478605786 talk.pdf — Mais qu'est-ce que je fous? - comment glander et apprendre les maths en même temps
1478605846 emacs@birdsong - talk.tex : /home/jeff/src/jma/talks/2016-11__devfest-que-je-fous/talk.tex
1478605906 talk.pdf — Mais qu'est-ce que je fous? - comment glander et apprendre les maths en même temps
1478605966 emacs@birdsong - macros.tex : /home/jeff/src/jma/talks/2016-11__devfest-que-je-fous/macros.tex
  \end{Verbatim}
\end{frame}

\begin{frame}[fragile]
  \frametitle{Les données}
  Et ça donne :
  \vspace{1cm}
  \begin{Verbatim}
1478605245 emacs@birdsong - talk.tex : /home/jeff/src/jma/talks/2016-11__devfest-que-je-fous/talk.tex
1478605305 emacs@birdsong - talk.tex : /home/jeff/src/jma/talks/2016-11__devfest-que-je-fous/talk.tex
1478605365 emacs@birdsong - talk.tex : /home/jeff/src/jma/talks/2016-11__devfest-que-je-fous/talk.tex
1478605425 talk.pdf — Mais qu'est-ce que je fous? - comment glander et apprendre les maths en même temps
1478605486 emacs@birdsong - talk.tex : /home/jeff/src/jma/talks/2016-11__devfest-que-je-fous/talk.tex
1478605546 emacs@birdsong - talk.tex : /home/jeff/src/jma/talks/2016-11__devfest-que-je-fous/talk.tex
1478605606 emacs@birdsong - talk.tex : /home/jeff/src/jma/talks/2016-11__devfest-que-je-fous/talk.tex
1478605666 emacs@birdsong - talk.tex : /home/jeff/src/jma/talks/2016-11__devfest-que-je-fous/talk.tex
1478605726 emacs@birdsong - talk.tex : /home/jeff/src/jma/talks/2016-11__devfest-que-je-fous/talk.tex
1478605786 talk.pdf — Mais qu'est-ce que je fous? - comment glander et apprendre les maths en même temps
1478605846 emacs@birdsong - talk.tex : /home/jeff/src/jma/talks/2016-11__devfest-que-je-fous/talk.tex
1478605906 talk.pdf — Mais qu'est-ce que je fous? - comment glander et apprendre les maths en même temps
1478605966 emacs@birdsong - macros.tex : /home/jeff/src/jma/talks/2016-11__devfest-que-je-fous/macros.tex
  \end{Verbatim}
\end{frame}

\begin{frame}
  \phrase{Quelles questions puis-je poser?}
\end{frame}

\begin{frame}
  \phrase{Machine Learning}
\end{frame}

\begin{frame}
  \phrase{Data Science}
\end{frame}

\begin{frame}
  \phrase{Statistics}
\end{frame}

\begin{frame}
  \cimgggg{boxplot-vs-pdf.png}
\end{frame}

\begin{frame}
  \cimg{maths.jpg}
\end{frame}

\begin{frame}
  \frametitle{Maths}
  \phrase{Vector Space}
  \vspace{7mm}
  \phrase{Features}
  \vspace{7mm}
  \phrase{Feature engineering}
\end{frame}

\begin{frame}
  \frametitle{Activity by host}
  % ./plot_history.py --color-host
  % cp /tmp/gtd-activity.png gtd-history-by-host.png
  \cimgff{gtd-history-by-host.png}
\end{frame}

\begin{frame}
  \frametitle{Activity by host class}
  % ./plot_history.py --color-host-class
  % cp /tmp/gtd-activity.png gtd-history-by-host-class.png
  \cimgff{gtd-history-by-host-class.png}
\end{frame}

\begin{frame}
  \frametitle{Recent activity}
  % ./plot_recent_days.py
  % cp /tmp/gtd-activity.png gtd-recent-days-10.png
  % ./plot_recent_days.py --num-days 20
  % cp /tmp/gtd-activity.png gtd-recent-days-20.png
  \only<1>{\cimgff{gtd-recent-days-10.png}}
  \only<2>{\cimgff{gtd-recent-days-20.png}}
  % Note that does not include tablet use.
  % Talk about seeing weekends.
  % And talk about discovering long tea breaks
\end{frame}

\begin{frame}
  \frametitle{Pauses}
  \only<1>{\cimgff{gtd-pauses-30.png}}
  \only<2>{\cimgff{gtd-pauses-60.png}}
  \only<3>{\cimgff{gtd-pauses-90.png}}
  \only<4>{\cimgff{gtd-pauses-120.png}}
  % Note that I'm not measuring what I do on a tablets.
  % These pause plots are a bit old, but they stay sadly uninformative.
\end{frame}

\begin{frame}
  \frametitle{Bag of Words}
  \phrase{sac de mots}
\end{frame}

\begin{frame}[t]
  \frametitle{Bag of Words}
  \vspace{1cm}
  \only<1>{
    \tt
    {\ }\\
    Le chat est orange. \\
    Le chien court vite. \\
    {\ }
  }
  \only<2>{
    \tt
    6\ \ \ 1\ \ \ \ 7\ \ \ \  2\\
    Le chat est orange. \\
    Le chien court vite. \\
    6\ \ \ \ 3\ \ \ \ \ 4\ \ \ \  5
  }
  \only<3>{
    \tt
    6\ \ \ 1\ \ \ \ 7\ \ \ \  2\\
    \gray{Le chat est orange.} \\
    \gray{Le chien court vite.} \\
    6\ \ \ \ 3\ \ \ \ \ 4\ \ \ \  5
  }
  \only<4>{
    \tt
       [[6, 1, 7, 2], \\
         {\ }\\
         {\ }\\
       {}\ [6, 3, 4, 5]]
  }
  \only<5>{
    \tt
       \gray{[[6, 1, 7, 2]} \\
       {\ }[1, 1, 0, 0, 0, 1, 1] \\
       {\ }[0, 0, 1, 1, 1, 1, 0] \\
       \gray{{}\ [6, 3, 4, 5]]}
  }
  \only<6>{
    \tt
       {\ }\\
       {\ }[1, 1, 0, 0, 0, 1, 1] \\
       {\ }[0, 0, 1, 1, 1, 1, 0] \\
       {\ }
  }
  \only<7>{
    \tt
       \gray{Le chat est orange.}\\
       {\ }[1, 1, 0, 0, 0, 1, 1] \\
       {\ }[0, 0, 1, 1, 1, 1, 0] \\
       \gray{Le chien court vite.}
  }
\end{frame}

\begin{frame}
  \frametitle{Cosine Similarity}
  \only<1-2>{
  \begin{minipage}{.4\textwidth}
  \input{dotproduct.pdf_t}
  \end{minipage}%
  \begin{minipage}{.6\textwidth}
    \only<1>{
    \begin{displaymath}
      \cos\,\theta = \frac{u\cdot v}{\parallel u\parallel \parallel v\parallel}
    \end{displaymath}
    }
    \only<2>{
    \begin{displaymath}
      \cos\,\theta = u\cdot v
    \end{displaymath}

    \bigskip
    \centerline{(if $u$ and $v$ have norm 1)}
    }
  \end{minipage}
  }
  \only<3>{
    \tt
       \gray{Le chat est orange.}\\
       {\ }[1, 1, 0, 0, 0, 1, 1] \\
       {\ }[0, 0, 1, 1, 1, 1, 0] \\
       \gray{Le chien court vite.}
  }
  \only<4>{
    \tt
    {\.}\\
    $u =$[1, 1, 0, 0, 0, 1, 1] \\
    $v =$[0, 0, 1, 1, 1, 1, 0] \\
    $u\cdot v = 0 + 0 + 0 + 0 + 0 + 1 + 0 = 1$

    \begin{displaymath}
      \cos\theta = \frac{u\cdot v}{\parallel u\parallel \parallel v\parallel}
      = \frac{1}{\sqrt{4}\cdot \sqrt{4}}
      = \frac{1}{4}
    \end{displaymath}
  }
\end{frame}

\begin{frame}
  \frametitle{Hamming Distance}
    \tt
    {\.}\\
    \hspace{18mm}$u =$[1, 1, 0, 0, 0, \red{1}, 1] \\
    \hspace{18mm}$v =$[0, 0, 1, 1, 1, \red{1}, 0] \\[5mm]
    $H(u,v) = 0 + 0 + 0 + 0 + 0 + \red{1} + 0$

\end{frame}

\begin{frame}
  \frametitle{Jaccard Index}

  \begin{displaymath}
    J(A,B) = \frac{\left| A\cap B\right|}{\left| A\cup B\right|
    }
  \end{displaymath}
\end{frame}

\begin{frame}
  \frametitle{Bigrams}
  \only<1>{\phrase{more context}}
  \only<2>{
    Le chat est orange.\\
    Le chien court vite.
  }
  \only<3>{
    \gray{Le chat est orange.}\\
    \gray{Le chien court vite.}\\[5mm]
    \{ le, chat, est, orange, chien, court, vite, \\
    \ \ le chat, chat est, est orange, \\
    \ \ le chien, chien court, court vite \}
  }
\end{frame}

\begin{frame}[t]
  \frametitle{Bag of Words}
  \vspace{1cm}
  Exemple :

  \Large
  \begin{quote}
    Il est nuit. La cabane est pauvre, mais bien close.\\
    Le logis est plein d'ombre et l'on sent quelque chose\\
    Qui rayonne à travers ce crépuscule obscur.\\
    Des filets de pêcheur sont accrochés au mur.\\
    Au fond, dans l'encoignure où quelque humble vaisselle\\
    Aux planches d'un bahut vaguement étincelle,\\
    On distingue un grand lit aux longs rideaux tombants.\\
    Tout près, un matelas s'étend sur de vieux bancs,\\
    Et cinq petits enfants, nid d'âmes, y sommeillent\\
    La haute cheminée où quelques flammes veillent\\
    Rougit le plafond sombre, et, le front sur le lit,\\
    Une femme à genoux prie, et songe, et pâlit.\\
    C'est la mère. Elle est seule. Et dehors, blanc d'écume,\\
    Au ciel, aux vents, aux rocs, à la nuit, à la brume,\\
    Le sinistre océan jette son noir sanglot.
  \end{quote}
\end{frame}

\begin{frame}[t]
  \frametitle{Bag of Words}
  \vspace{1cm}
  \only<1-2>{
    Exemple (plus simple) :
    
    \begin{quote}
      Il est nuit. La cabane est pauvre, mais bien close.\\
      Le logis est plein d'ombre et l'on sent quelque chose\\
    \end{quote}
  }
  \only<2>{
    \lstinputlisting[language=python]{src/pauvres_gens1.py}
  }
  \only<3>{
    \lstinputlisting[language=python, firstline=7]{src/pauvres_gens2.py}
  }
  \only<4>{
    \Large
   Pluie ou bourrasque, il faut qu'il sorte, il faut qu'il aille,\\
   Il n'avait pas assez de peine ; il faut que j'aille\\[5mm]

   Pluie ou bourrasque, il faut qu'il sorte, il faut qu'il aille,\\
   Quand il verra qu'il faut nourrir avec les nôtres\\[5mm]

   l s'en va dans l'abîme et s'en va dans la nuit.\\
   Or, la nuit, dans l'ondée et la brume, en décembre,\\[5mm]

   Comme il faut calculer la marée et le vent !\\
   Et l'onde et la marée et le vent en colère.\\[5mm]

   C'est l'heure où, gai danseur, minuit rit et folâtre\\
   Et c'est l'heure où minuit, brigand mystérieux,\\[5mm]

   Sous sa cape aux longs plis qu'est-ce donc qu'elle emporte ?\\
   Qu'est-ce donc qu'elle cache avec un air troublé
  }
  \only<5>{
    \lstinputlisting[language=python, firstline=7]{src/pauvres_gens3.py}
  }
  \only<6>{
    \Large
    \texttt{'Comme il faut calculer la marée et le vent !'}\\[10mm]

    \texttt{ ['comme', 'il', 'faut', 'calculer', 'la', 'marée', 'et',
        'le', 'vent', 'comme il', 'il faut', 'faut calculer',
        'calculer la', 'la marée', 'marée et', 'et le', 'le vent']}
    }
  \only<7>{
    \Large
   Pluie ou bourrasque, il faut qu'il sorte, il faut qu'il aille,\\
   Quand il verra qu'il faut nourrir avec les nôtres\\[5mm]

   Comme il faut calculer la marée et le vent !\\
   Et l'onde et la marée et le vent en colère.\\[5mm]

   Qu'est-ce donc que Jeannie a fait chez cette morte ?\\
   Qu'est-ce donc que Jeannie emporte en s'en allant ?\\[5mm]

   Sous sa cape aux longs plis qu'est-ce donc qu'elle emporte ?\\
   Qu'est-ce donc qu'elle cache avec un air troublé
  }
\end{frame}

\begin{frame}
  \frametitle{TF - IDF}
  \blue{
    \begin{displaymath}
      TF_{td} = \frac{f_{td}}{\max_k f_{kd}} \qquad\qquad
      IDF_t = \log_2\left( \frac{N}{n_t} \right)
    \end{displaymath}
  }
  \blue{
    \begin{displaymath}
      TF\mbox{-}IDF_{td} = TF_{td} \cdot IDF_t
    \end{displaymath}
  }

  with
  \begin{align*}
    f_{td} &= \mbox{frequency of word (term) $t$ in document $d$} \\
    N &= \mbox{number of documents}\\
    n_t &= \mbox{number of documents containing term $t$}
  \end{align*}
\end{frame}

\begin{frame}
  \frametitle{$k$-means}
  \only<1>{\phrase{$k$-means}}
  \only<2>{\cimgggg{cluster-1.png}}
  \only<3>{\cimg{cluster-2.png}}
  \only<4>{\cimg{cluster-3.png}}
  \only<5>{\cimgg{cluster-4.png}}
\end{frame}

\begin{frame}
  \frametitle{$k$ Nearest Neighbours}
  \only<1>{\cimgff{plot-similar-1.png}}
  \only<2>{\cimgff{plot-similar-2.png}}
  \only<3>{\cimgff{plot-similar-3.png}}
  \only<4>{\cimgff{plot-similar-4.png}}
  \only<5>{\input{knn.pdf_t}}
\end{frame}

\begin{frame}
  
\end{frame}

\begin{frame}
  
\end{frame}

\begin{frame}
  
\end{frame}

\begin{frame}
  
\end{frame}

% ======================================================================
% End

\begin{frame}
  \frametitle{Resources}
  \cimggg{nmlm.png}

  \vspace{5mm}
  \centerline{\url{http://www.meetup.com/Nantes-Machine-Learning-Meetup/}}
\end{frame}

\begin{frame}
  \frametitle{Resources}
  \cimggg{ml-week.png}

  \vspace{5mm}
  \centerline{\url{http://www.ml-week.com/}}
\end{frame}

\begin{frame}
  \phrase{Questions?}
\end{frame}

\end{document}
